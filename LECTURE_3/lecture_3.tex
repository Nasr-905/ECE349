\chapter{Superposition and Power}
\section{Superposition}
\begin{theorem}
    [Superposition]
    The response of a linear circuit to a sum of sources is the sum of the responses to each source acting alone, \textit{even if the sources have different frequencies}.
\end{theorem}

\begin{example}
    INSERT IMAGE FROM PHONE\\
    Solve at $100 rad/s$
    \begin{align}
        \underline{I_1} = \frac{100 \angle{0}}{100 + j100} = 0.707 \angle{-45^\circ} \\
        i_1(t) = 0.707\cos(100t - 45^\circ)
    \end{align}
    INSERT IMAGE FROM PHONE\\
    Solve at $200 rad/s$
    \begin{align}
        \underline{I_2} = \frac{100 \angle{0}}{100 + j200}\times 2\angle{0} = 0.894\angle{-63^\circ} \\
        i_2(t) = 0.894\cos(200t - 63^\circ)
    \end{align}
    INSERT IMAGE FROM PHONE\\
    Super impose in the time-domain. \\
    \begin{align}
        i(t) & = i_1(t) + i_2(t)                                         \\
             & = 0.707\cos(100t - 45^\circ) + 0.894\cos(200t - 63^\circ)
    \end{align}
\end{example}

\section{Periodic Non-Sinusoidal Systems}
\begin{remark}
    Periodic stimuli can be converted into sinusoidal stimuli using \textbf{Fourier analysis.}
\end{remark}
\begin{example}
    [Square Wave]
    \begin{align}
        v_s(t) & = a_0 \sum a_k \cos(k\omega t) + \sum b_k \sin(k\omega t) \\
        a_0    & = \frac{1}{T}\int_{0}^{T}v_s(t)dt                         \\
        a_k    & = \frac{2}{T}\int_{0}^{T}v_s(t)\cos(k\omega t)dt          \\
        b_k    & = \frac{2}{T}\int_{0}^{T}v_s(t)\sin(k\omega t)dt
    \end{align}
\end{example}
\begin{figure}[h]
    \centering
    \begin{tikzpicture}
        \draw[->] (0,0) -- (5,0) node[right] {$t$};
        \draw[->] (0,0) -- (0,2) node[above] {$V$};
        \draw[thick] (0,1) -- (1,1) -- (1,-1) -- (2,-1) -- (2,1) -- (3,1) -- (3,-1) -- (4,-1) -- (4,1) -- (5,1);
    \end{tikzpicture}
    \caption{Square wave}
\end{figure}

\begin{example}
    [Square Wave 2]
    Say we wanted to solve \ref{fig:Square wave circuit} with a square wave source. We can use superposition of an infinite number of sinusoidal sources to solve this circuit.
\end{example}
% tikz circuit with square source, resistor and inductor
\begin{figure}[h]
    \centering
    \begin{circuitikz}
        \draw (0,0) to[sV, l=$Square wave$] (0,2) to[R, l=$R$] (2,2) to[L, l=$L$] (2,0) -- (0,0);
    \end{circuitikz}
    \caption{Square wave circuit}
\end{figure}

\section{Power}
\begin{definition}
    [Instantaneous Power]
    The instantaneous power is the product of the instantaneous voltage and current.
    \[
        p(t) = v(t)i(t)
    \]
    AC source on resistor
\end{definition}
\begin{example}
    [Power in resistor]
    \begin{align}
        v(t) & = V\cos(\omega t)                                                   \\
        i(t) & = I\cos(\omega t)                                                   \\
        p(t) & = V\cos(\omega t)I\cos(\omega t) = \frac{VI}{2}(1 +\cos(2\omega t))
    \end{align}
\end{example}
\begin{figure}[h]
    \centering
    \begin{circuitikz}
        \draw (0,0) to[sV, l=$\hat{V}\cos{\omega t}$] (0,2) to[R, l=$R$] (2,2) to[short, l=$\hat{I}\cos{\omega t}$] (2,0) -- (0,0);
    \end{circuitikz}
    \caption{Power in resistor}
\end{figure}

\begin{definition}
    [Average Power]
    The average power is the time average of the instantaneous power.
    \[
        P = \frac{1}{T}\int_{0}^{T}p(t)dt
    \]
\end{definition}
% Tikz graph showing sinuoidal source and average power
\begin{definition}
    [Rective Power]
    The reactive power is the imaginary part of the complex power.
\end{definition}

\begin{remark}
    $\hat{I}, \hat{V}$ are phasors. The peak values of the sinusoidal source.
\end{remark}

let $\phi = \angle{V} - \angle{I}$

\begin{example}
    [Reactive Power]
    \begin{align}
        p(t) & = \hat{V}\cos(\omega t)\hat{I}\cos(\omega t - \phi)                                                             \\
             & = \frac{\hat{V}\hat{I}}{2}(\cos(\phi) + \cos(2\omega t + \phi))                                                 \\
             & = \frac{\hat{V}\hat{I}}{2}\cos(\phi(1 + \cos(2\omega t)) - \frac{\hat{V}\hat{I}}{2}\sin(\phi)(\sin(2\omega t))) \\
    \end{align}
    We get less real power if $\hat{V}$ and $\hat{I}$ are out of phase.
\end{example}